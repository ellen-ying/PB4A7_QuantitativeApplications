% Options for packages loaded elsewhere
\PassOptionsToPackage{unicode}{hyperref}
\PassOptionsToPackage{hyphens}{url}
%
\documentclass[
  11pt,
]{article}
\usepackage{amsmath,amssymb}
\usepackage{lmodern}
\usepackage{iftex}
\ifPDFTeX
  \usepackage[T1]{fontenc}
  \usepackage[utf8]{inputenc}
  \usepackage{textcomp} % provide euro and other symbols
\else % if luatex or xetex
  \usepackage{unicode-math}
  \defaultfontfeatures{Scale=MatchLowercase}
  \defaultfontfeatures[\rmfamily]{Ligatures=TeX,Scale=1}
  \setmainfont[]{Times New Roman}
\fi
% Use upquote if available, for straight quotes in verbatim environments
\IfFileExists{upquote.sty}{\usepackage{upquote}}{}
\IfFileExists{microtype.sty}{% use microtype if available
  \usepackage[]{microtype}
  \UseMicrotypeSet[protrusion]{basicmath} % disable protrusion for tt fonts
}{}
\makeatletter
\@ifundefined{KOMAClassName}{% if non-KOMA class
  \IfFileExists{parskip.sty}{%
    \usepackage{parskip}
  }{% else
    \setlength{\parindent}{0pt}
    \setlength{\parskip}{6pt plus 2pt minus 1pt}}
}{% if KOMA class
  \KOMAoptions{parskip=half}}
\makeatother
\usepackage{xcolor}
\usepackage[margin=1.0in]{geometry}
\usepackage{graphicx}
\makeatletter
\def\maxwidth{\ifdim\Gin@nat@width>\linewidth\linewidth\else\Gin@nat@width\fi}
\def\maxheight{\ifdim\Gin@nat@height>\textheight\textheight\else\Gin@nat@height\fi}
\makeatother
% Scale images if necessary, so that they will not overflow the page
% margins by default, and it is still possible to overwrite the defaults
% using explicit options in \includegraphics[width, height, ...]{}
\setkeys{Gin}{width=\maxwidth,height=\maxheight,keepaspectratio}
% Set default figure placement to htbp
\makeatletter
\def\fps@figure{htbp}
\makeatother
\setlength{\emergencystretch}{3em} % prevent overfull lines
\providecommand{\tightlist}{%
  \setlength{\itemsep}{0pt}\setlength{\parskip}{0pt}}
\setcounter{secnumdepth}{5}
\newcommand{\bcenter}{\begin{center}}
\newcommand{\ecenter}{\end{center}}
\newcommand{\btitlepage}{\begin{titlepage}}
\newcommand{\etitlepage}{\end{titlepage}}
\usepackage{setspace}\onehalfspacing
\usepackage{booktabs}
\usepackage[font=small,labelfont=bf]{caption}
\usepackage{booktabs}
\usepackage{longtable}
\usepackage{array}
\usepackage{multirow}
\usepackage{wrapfig}
\usepackage{float}
\usepackage{colortbl}
\usepackage{pdflscape}
\usepackage{tabu}
\usepackage{threeparttable}
\usepackage{threeparttablex}
\usepackage[normalem]{ulem}
\usepackage{makecell}
\usepackage{xcolor}
\ifLuaTeX
  \usepackage{selnolig}  % disable illegal ligatures
\fi
\IfFileExists{bookmark.sty}{\usepackage{bookmark}}{\usepackage{hyperref}}
\IfFileExists{xurl.sty}{\usepackage{xurl}}{} % add URL line breaks if available
\urlstyle{same} % disable monospaced font for URLs
\hypersetup{
  hidelinks,
  pdfcreator={LaTeX via pandoc}}

\author{}
\date{\vspace{-2.5em}}

\begin{document}

\begin{titlepage}

\begin{center}

\vspace*{30mm}

Candidate number: 49045

\vspace*{5mm}

\hypertarget{replication-of-hansen-2015-punishment-and-deterrence-evidence-from-drunk-driving}{%
\section*{Replication of Hansen (2015) Punishment and Deterrence:
Evidence from Drunk
Driving}\label{replication-of-hansen-2015-punishment-and-deterrence-evidence-from-drunk-driving}}
\addcontentsline{toc}{section}{Replication of Hansen (2015) Punishment
and Deterrence: Evidence from Drunk Driving}

\vspace*{30mm}

Submitted as the summative assessment for\\

PB4A7: Quantitative Applications for Behavioural Science 2022

\end{center}

\end{titlepage}

\newpage

In the study \emph{Punishment and Deterrence: Evidence from Drunk
Driving} \textbf{(Hansen, 2015)}, Hansen investigated the effect of
punishments and sanctions on reducing repeat drunk driving. The study
implemented a quasi-experimental design, utilising the administrative
records from 1995 to 2011 of the state of Washington, U.S., where two
thresholds of blood alcohol content (BAC) are used to determine the
status of driving under the influence (DUI). Specifically, a driver with
a measured BAC over 0.08 is considered a case of DUI and will be
punished via measures such as fines, jail time, and driving license
suspension. One with a BAC over 0.15 is considered a case of aggravated
DUI, to which more severe punishments and sanctions are applied. Since
BAC measure has clear-cut numeric thresholds for determining whether a
driver will receive harsher punishments, and neither drivers or police
can manipulate this measure, Hansen applied a regression discontinuity
design to analyse the data \textbf{(expand on the justification of using
RDD)}. He hypothesized that receiving harsher punishments and sanctions
at both thresholds would reduce offenders' future recidivism of DUI. He
further applied the RDD to analyse the effect of receiving harsher
punishments on the degree of deterrence, incapacitation, and
rehabilitation in order to identify the mechanism through which harsher
punishments might reduce recidivism \textbf{(check the paper to see
whether this sentence is precise)}.

\textbf{Ideally also the results}

The present study aims to replicate the findings from \textbf{Hansen
(2015)} using regression discontinuity design applied to a similar data
\textbf{(what is it exactly\ldots)}. It will be focused on estimating
the effect of receiving punishments on reducing recidivism at the 0.08
BAC threshold only. The paper proceeds as follows. Section 1 discusses
the econometric method and assumptions underlying its application to the
current data. Section 2 presents the main results, and Section 3
discusses critiques and extensions to the original study. Section 4
concludes.

\hypertarget{methods-and-assumptions}{%
\section{Methods and Assumptions}\label{methods-and-assumptions}}

\hypertarget{assumptions-of-the-regression-discontinuity-design}{%
\subsection{Assumptions of the regression discontinuity
design}\label{assumptions-of-the-regression-discontinuity-design}}

The present study applies the regression discontinuity design to the
data provided by the class instructor. Several assumptions need to be
met so that the regression discontinuity design can give accurate
estimates. First, people need to be randomly assigned to receiving
punishment at the BAC threshold. In other words, neither the drivers nor
the police can manipulate the BAC level and thus whether one will
receive punishment. \textbf{Hansen (2015)} has already given a plausible
theoretical justification for this assumption. Empirically, I plot the
distribution of BAC and test for discontinuity in its distribution at
the threshold. Figure \ref{fig:bac_hist_continuous} shows the
distribution of BAC level. Based on a recently developed local
polynomial density estimator \textbf{(CJM, 2020)}, the hypothesis that
the distribution is continuous at 0.08 cannot be rejected at the level
of 0.05 (\emph{p} = 0.890). Therefore, there is no evidence for the
existence of manipulation on the BAC level.

\begin{figure}[H]
  \centering
  \includegraphics[width=0.9\columnwidth]{../figures/bac_histogram_continuous.pdf}
  \caption{\textbf{Histogram of blood alcohol content as a continuous variable.} The blood alcohol content is plotted as a continuous variable with a bin width of 0.001, the precision used on the breathalysers. The y-axis represents the frequency of observations in each bin. The vertical black lines represent the two thresholds at 0.08 and 0.15.}
  \label{fig:bac_hist_continuous}
\end{figure}

The second assumption is that the running variable should not contain
non-random heaping, which can lead to biased estimates in regression
discontinuity models \textbf{(Barreca et al., 2011)}. In other words,
BAC should not be much more likely to take certain values than others.
Based on Figure \ref{fig:bac_hist_continuous}, this assumption seems to
be violated. Curiously, when BAC is plotted as a discrete variable,
non-random heaping disappears (Figure \ref{fig:bac_hist_discrete}). A
closer inspection of the data suggests that this is probably due to the
lack of precision in BAC in the current data. Many BAC values deviate by
a very small amount from the values that are supposed to be given by the
breathalysers (i.e., precise to three digits after the decimal point).
Some bins have a value at the left boundary deviating upwards and a
value at the right boundary deviating downwards. Such bins will contain
a larger number of observations than other bins and thus leads to
heaping. This would not occur if the data records the values precisely
so that the number of observations are rightly separated into and
plotted by two bins, or if we treat BAC as discrete and give each value
its own bin. \textbf{(consider using a simulation to justify this
claim?)}

\begin{figure}[H]
  \centering
  \includegraphics[width=0.9\columnwidth]{../figures/bac_histogram_discrete.pdf}
  \caption{\textbf{Histogram of blood alcohol content as a discrete variable.} The blood alcohol content is plotted as a discrete variable with a bin width of 0.001, the precision used on the breathalysers. The y-axis represents the frequency of observations in each bin. The vertical black lines represent the two thresholds at 0.08 and 0.15.}
  \label{fig:bac_hist_discrete}
\end{figure}

\hypertarget{the-model}{%
\subsection{The model}\label{the-model}}

The present study utilises a local linear regression model with a
rectangular kernel weight function to estimate the effect of receiving
punishments at the threshold on recidivism. For sensitivity analyses,
the models are re-estimated using second-order polynomials and triangle
kernel function). The model is specified by the following function:

\begin{equation}
  \label{eqn:model}
  R_i = \alpha + \beta_1 DUI_i + \beta_2 BAC_i + \beta_3 DUI_i \times BAC_i + \tau Z_i + \epsilon_i
\end{equation}

where the variable \(DUI_i\) is an indicator of whether BAC is above the
0.08 threshold, \(BAC_i\) is the measure of BAC level, \(R_i\) is an
indicator of recidivism, and \(Z_i\) is a vector of control variables.
The variable \(BAC_i\) in the model is centered around the threshold
value of 0.08 so that the coefficients directly reflect the estimates of
the effect.

\hypertarget{inclusion-of-control-variables}{%
\subsection{Inclusion of control
variables}\label{inclusion-of-control-variables}}

In order to determine whether any covariates should be included in the
model, I run preliminary analyses on the effect of receiving punishments
at the threshold on four predetermined characteristics, including three
demographic variables (gender, race, and age) and the BAC test being
conducted in a traffic accident. The analyses use the same local linear
regression model specified in equation (\ref{eqn:model}), with a
bandwidth of 0.05 and no kernel weight function. As Table
\ref{tab:covariate} shows, I fail to reject any of the null hypotheses
that predetermined characteristics remained the same at the threshold.
This indicates that gender, race, age, and the BAC test being conducted
in a traffic accident, on average, were not different between people who
did not receive punishments and those who received punishments at the
threshold
\footnote{This is an additional evidence for the absence of manipulation at the threshold, which further assures the regression discontinuity design can deliver unbiased estimates.}.
According to \textbf{Calonico et al.~(2019)}, I will include the four
predetermined characteristics in regression discontinuity models to
improve the estimation precision of the effects of receiving punishments
at the threshold on recidivism.

\begingroup
\renewcommand{\arraystretch}{1.3}

\begin{table}

\caption{Regression Discontinuity Estimates of the Effect of Receiving Punishments at the 0.08 BAC Threshold on Predetermined Characteristics}
\label{tab:covariate}
\centering
\begin{threeparttable}
\begin{tabular}[t]{l>{\centering\arraybackslash}p{5em}>{\centering\arraybackslash}p{5em}>{\centering\arraybackslash}p{5em}>{\centering\arraybackslash}p{5em}}
\toprule
  & Male & White & Age & Accident\\
\midrule
\textit{DUI} & 0.006 & 0.006 & –0.141 & –0.003\\
 & (0.006) & (0.005) & (0.164) & (0.004)\\
Mean & 0.784 & 0.846 & 0.085 & 33.99\\
Num. of Obs. & 89,967 & 89,967 & 89,967 & 89,967\\
\bottomrule
\end{tabular}
\begin{tablenotes}
\small
\item \textit{Note.} Regression discontinuity based estimates of the effect of receiving punishments at the 0.08 BAC threshold on four predetermined characteristics. All models use a bandwidth of 0.05 and a rectangular kernel weight function. Counterfactual predictions of mean recidivism is calculated at the 0.079 BAC threshold. Heteroscedasticity-robust standard errors are in parentheses. $^{*}\, p<0.1$, $^{**}\, p<0.05$, $^{***}\, p<0.01$.
\end{tablenotes}
\end{threeparttable}
\end{table}

\endgroup

\hypertarget{results}{%
\section{Results}\label{results}}

\hypertarget{the-effect-of-punishments-on-recidivism}{%
\subsection{The Effect of Punishments on
Recidivism}\label{the-effect-of-punishments-on-recidivism}}

Table \ref{tab:main} reports the estimated effect of receiving
punishments at the threshold on recidivism. Local linear regression
model with a rectangular kernel function gives the estimate that
receiving punishments at the 0.08 BAC threshold decreases recidivism by
2.4 percentage points, which is statistically significant at the level
of 0.01. Local second-order polynomial with a rectangular kernel
function gives an estimate of a decrease in recidivism by 1.4 percentage
points, which is statistically significant at the level of 0.05. These
estimates are consistent across both bandwidths and across models with
different types of kernel functions.

\begingroup
\renewcommand{\arraystretch}{1.1}

\begin{table}

\caption{Regression Discontinuity Estimates of the Effect of Receiving Punishments at the 0.08 BAC Threshold on Recidivism}
\label{tab:main}
\centering
\begin{threeparttable}
\begin{tabular}[t]{l>{\centering\arraybackslash}p{8em}>{\centering\arraybackslash}p{8em}>{\centering\arraybackslash}p{8em}>{\centering\arraybackslash}p{8em}}
\toprule
\multicolumn{1}{c}{ } & \multicolumn{2}{c}{Rectangular kernel} & \multicolumn{2}{c}{Triangular kernel} \\
\cmidrule(l{3pt}r{3pt}){2-3} \cmidrule(l{3pt}r{3pt}){4-5}
  & Linear & Quadratic & Linear & Quadratic\\
\midrule
\multicolumn{5}{l}{\textit{A. DUI $\in$ [0.03, 0.13]}} \\
\textit{DUI} & –0.024*** & –0.014** & –0.020*** & –0.014**\\
 & (0.004) & (0.006) & (0.005) & (0.006)\\
Mean & 0.104 & 0.099 & 0.100 & 0.099\\
Controls & Yes & Yes & Yes & Yes\\
Num. of Obs. & 89,967 & 89,967 & 89,967 & 89,967\\
\addlinespace
\multicolumn{5}{l}{\textit{B. DUI $\in$ [0.055, 0.105]}} \\
\textit{DUI} & –0.021*** & –0.014* & –0.018*** & –0.016*\\
 & (0.006) & (0.008) & (0.006) & (0.009)\\
Mean & 0.101 & 0.098 & 0.101 & 0.100\\
Controls & Yes & Yes & Yes & Yes\\
Num. of Obs. & 46,957 & 46,957 & 46,957 & 46,957\\
\bottomrule
\end{tabular}
\begin{tablenotes}
\small
\item \textit{Note.} Regression discontinuity based estimates of the effect of receiving punishments at the 0.08 BAC threshold on recidivism. Panel A presents estimates based on a 0.05 bandwidth, and Panel B presents estimates based on a 0.025 bandwidth. The table includes results from both linear and quadratic models, with either a rectangular or a triangular kernel weight function. Controls include individuals' gender, race, age, and an indicator of whether the BAC test was conducted in a traffic accident. Counterfactual predictions of mean recidivism are calculated at the 0.079 BAC threshold and mean age of the respective populations, averaging over individuals' gender and race, as well as over whether the BAC testing was conducted in an accident. Heteroscedasticity-robust standard errors are in parentheses. $^{*}\, p<0.1$, $^{**}\, p<0.05$, $^{***}\, p<0.01$.
\end{tablenotes}
\end{threeparttable}
\end{table}

\endgroup

Figure \ref{fig:rdplot} plots means of recidivism in bins and predicted
recidivism based on linear regression models or second-order polynomials
using observations within the interval \(BAC \in [0.06, 0.11]\). Panel A
and B and Panel C and D use different methods for binning observations
and representing confidence intervals to ensure the robustness of the
visual evidence (See the figure caption). All panels show an apparent
drop in recidivism at the BAC threshold of 0.08. This visually indicates
that there is an effect of receiving punishments at the BAC threshold in
decreasing recidivism.

\begin{figure}[H]
  \centering
  \includegraphics[width=1\columnwidth]{../figures/combined.pdf}
  \caption{\textbf{Regression Discontinuity Plot of the Effect of Receiving Punishments at the 0.08 BAC Threshold on Recidivism.} Plot of means of recidivism in bins and predicted recidivism using simple regression models. In plotting the binned means (gray points), Panel A and B choose the number of bins on each side of the threshold using the formula $min\{\sqrt{n}, 10 \times \frac{\ln{n}}{\ln{10}}\}$, while Panel C and D choose the number using quantile spaced binning **(Calonico et al., 2015)**. For the confidence intervals (shaded areas in red), Panel A and B plot the 95\% confidence interval of the regression models, while Panel C and D plot the 95\% confidence interval of each bin. For the regression models (red lines), Panel A and C plot the best fitted linear model, while Panel B and D plot the best fitted second-order polynomial. All panels plot the data within the interval $BAC \in [0.06, 0.11]$. The vertical dashed lines represent the BAC threshold at 0.08.}
  \label{fig:rdplot}
\end{figure}

\hypertarget{robustness}{%
\subsection{Robustness}\label{robustness}}

Since there seems to be non-random heaping in the current data (Figure
\ref{fig:bac_hist_continuous}), this section will use donut regression
discontinuity models to investigate the robustness of the results
presented in the last section. Donut regression discontinuity models
entirely drop the observations near the threshold. Under appropriate
assumptions, they are effective in preventing a heap just at the
threshold from biasing the estimates \textbf{(Barreca et al., 2011)}.

For the present study, I drop the observations in the interval
\(BAC \in [0.079, 0.081]\), and re-estimate local linear and quadratic
models with either a rectangular or a triangular kernel weight function.
It is worth stressing that robustness test using local polynomials is
especially important in a donut regression discontinuity design. This is
because a donut regression model gets rid of the very observations on
which the estimates of local average effects rely upon. The estimates
given by donut regression models are essentially based on models'
extrapolation over the region of the donut \textbf{(Dowd, 2021)} and may
be especially sensitive to the assumptions about the underlying
functional form \textbf{(Consider justifying this using a simulation)}.
Using polynomials with different orders can ensure the robustness of the
results over different functional assumptions.

The results from donut regression discontinuity models are presented in
Table \ref{tab:donut}. According to the local linear regression model
with a rectangular kernel function, receiving punishments at the
threshold decreases recidivism by 2.6 percentage points and is
statistically significant at the level of 0.01. According to the local
second-order polynomial with a rectangular kernel function, receiving
punishments at the threshold decreases recidivism by 1.4 percentage
points and is statistically significant at the level of 0.1. These
estimates are consistent across both bandwidths and across models with
different types of kernel functions, except for not being statistically
significant in quadratic models estimated with narrower bandwidth of
0.025. Therefore, the results given by donut regression discontinuity
models are essentially identical with those given by ordinary models
presents in the last section.

\begingroup
\renewcommand{\arraystretch}{1.1}

\begin{table}

\caption{Donut Regression Discontinuity Estimates of the Effect of Receiving Punishments at the 0.08 BAC Threshold on Recidivism}
\label{tab:donut}
\centering
\begin{threeparttable}
\begin{tabular}[t]{l>{\centering\arraybackslash}p{8em}>{\centering\arraybackslash}p{8em}>{\centering\arraybackslash}p{8em}>{\centering\arraybackslash}p{8em}}
\toprule
\multicolumn{1}{c}{ } & \multicolumn{2}{c}{Rectangular kernel} & \multicolumn{2}{c}{Triangular kernel} \\
\cmidrule(l{3pt}r{3pt}){2-3} \cmidrule(l{3pt}r{3pt}){4-5}
  & Linear & Quadratic & Linear & Quadratic\\
\midrule
\multicolumn{5}{l}{\textit{A. DUI $\in$ [0.03, 0.13]}} \\
\textit{DUI} & –0.026*** & –0.014* & –0.022*** & –0.018**\\
 & (0.005) & (0.007) & (0.005) & (0.006)\\
Mean & 0.105 & 0.099 & 0.102 & 0.101\\
Controls & Yes & Yes & Yes & Yes\\
Num. of Obs. & 88,085 & 88,085 & 88,085 & 88,085\\
\addlinespace
\multicolumn{5}{l}{\textit{B. DUI $\in$ [0.055, 0.105]}} \\
\textit{DUI} & –0.022*** & –0.015 & –0.020*** & –0.020\\
 & (0.007) & (0.011) & (0.007) & (0.012)\\
Mean & 0.103 & 0.099 & 0.103 & 0.104\\
Controls & Yes & Yes & Yes & Yes\\
Num. of Obs. & 45,075 & 45,075 & 45,075 & 45,075\\
\bottomrule
\end{tabular}
\begin{tablenotes}
\small
\item \textit{Note.} Regression discontinuity based estimates of the effect of receiving punishments at the 0.08 BAC threshold on recidivism after dropping observations in the interval [0.079, 0.081]. Panel A presents estimates based on a 0.05 bandwidth, and Panel B presents estimates based on a 0.025 bandwidth. The table includes results from both linear and quadratic models, with either a rectangular or a triangular kernel weight function. Controls include individuals' gender, race, age, and an indicator of whether the BAC test was conducted in a traffic accident. Counterfactual predictions of mean recidivism are calculated at the 0.079 BAC threshold and mean age of the respective populations, averaging over individuals' gender and race, as well as over whether the BAC testing was conducted in an accident. Heteroscedasticity-robust standard errors are in parentheses. $^{*}\, p<0.1$, $^{**}\, p<0.05$, $^{***}\, p<0.01$.
\end{tablenotes}
\end{threeparttable}
\end{table}

\endgroup

\hypertarget{discussion-critique-and-extension}{%
\section{Discussion, Critique and
Extension}\label{discussion-critique-and-extension}}

Using a regression discontinuity design, the present study aims to
replicate the findings of \textbf{Hansen (2015)} about the effect of
punishments and sanctions on reducing repeat drunk driving.
Specifically, the present study focuses on testing whether receiving
punishments at the 0.08 Blood Alcohol Content (BAC) threshold reduces
future recidivism of drunk driving with a local linear regression model.
The findings show that receiving punishments at the threshold reduces
recidivism by up to 2.4 percentage points
\footnote{The average estimated effect is 2.1 percentage points. The average estimate is calculated over models based on different bandwidths and with different kernel functions, assuming equal weights of estimates.}.
Robustness test based on local second-order polynomials show that
receiving punishments at the threshold reduces recidivism by up to 1.6
percentage points
\footnote{The average estimated effect is 1.5 percentage points.}. These
estimates are consistent with those given by models that take into
account non-random heaping in the data. The findings of the present
study successfully replicate the main results of \textbf{Hansen (2015)},
which gives an estimated decrease in recidivism of up to 2 percentage
points due to receiving punishments at the 0.08 BAC threshold. The
success of replication demonstrates the validity of the methodology and
results of the original study.

\textbf{Hansen (2015)} also presents related results that the present
study, due to its scope and the limitation of the data available, does
not replicate. As a part of the strengths of the original study, these
findings either establish the robustness of the main results under
different assumptions or reveal the heterogeneity of the effect of
receiving punishments.

In terms of the robustness of the main results, the original study
estimates regression discontinuity models using polynomials up to the
third order, three types of kernel functions, and based on a series of
bandwidths that extend beyond 0.05 and 0.025. The results show that the
point estimates are relatively consistent, except for those given by the
local second-order polynomial and those estimated based on very small
bandwidths.

The heterogeneity of the effects of punishments include the effects
among different populations and the effect on different types of
recidivism. First of all, the original study separates analyses for
drivers with and without prior drunk driving experience, in addition to
running regression models on the entire population of drivers. This
allows the original study to find that the effect of receiving
punishments at the threshold is more pronounced among drivers with
previous drunk driving experience (by about 4.6 percentage points, which
is 25.2 percent
\footnote{The estimate of percentgae points is calcultaed by taking the mean over the two estimates reported in Table 3 of Hansen (2015).}).
Moreover, the study conducts sensitivity analyses using different
measures of prior drunk driving experiences, including whether a driver
has taken BAC tests, whether a driver has a previous BAC over 0.08 (and
is considered a case of DUI), as well as whether a driver has a prior
conviction. Across all measures, regression models give point estimates
and confidence intervals that are largely similar to each other. This
demonstrates the robustness of the differential effects of receiving
punishments at the threshold on different populations of drivers.
Second, the original study conducts a detailed analyses of the effect of
receiving punishments on three types of recidivism, refusal of taking a
BAC test, and the likelihood of being stopped by a police. It finds that
receiving punishments decreases the likelihood of being stopped as well
as recidivism of all kinds. The effect is the largest on recidivism in
which a driver has a BAC values in \([0.08, 0.15]\).

Despite multiple strengths of the original study, it can still be
improved in several aspects. First, although sensitivity analyses using
different bandwidths in the original study establish the robustness of
the results, bandwidths of 0.05 and 0.025 still seem to be arbitrary
choices of the researcher. Bandwidth choice can be improved by using
data-driven methods with a chosen principle. For example, Mean Squared
Error (MSE) optimal bandwidth choice method is available and has been
recently improved with a bias correction of point estimate for better
inference \textbf{(Imbens and Kalyanaraman, 2012; Cattaneo and
Vazquez-Bare, 2016)}. Alternatively, when the goal is to construct
optimal confidence intervals for inference, one can choose to use a
bandwidth selection method that minimises coverage error (CE)
probability \textbf{(Calonico et al., 2022)}. Both MSE-optimal and
CE-optimal bandwidths are based on the data and constructed according to
an explicit desideratum, which makes the bandwidth choice more objective
and transparent. Future research can and should make full use of these
methods as they are readily available in packages of major statistical
software \textbf{(Calonico et al., 2015; Calonico et al., 2017)}.

Second, in light of the present study, local second-order polynomials
consistently give lower estimates than linear models. In fact, this is
also the finding from the robustness test of the original study
(Appendix Table 5 of Hansen, 2015). This suggests that the estimates of
the effect of receiving punishments at the threshold are somewhat
dependent on the assumptions of the effect's functional form. Although
it is difficult to pin down one reasonable functional assumption due to
the complex nature of the effect of punishments, future research should
give equal weights in presenting the results given by different models.
This is especially important for such an estimand as the effect of
punishments on repeat drunk driving that is closely associated with
policy making, since even a small difference in the estimate can lead to
big differences when scaled up.

Third, the use of linear models is undesirable when recidivism, the
outcome variable, is a binary variable. In simple words, this is because
a mathematical inconsistency exists between the estimand (which can only
take the value of 0 and 1) and the potential outcome of a linear model
(which can take the value of any real number). In practice, future
research can utilise logistic regressions in combination with regression
discontinuity design to estimate the effect of punishments on
recidivism. This should not be a difficult task because recent research
in econometrics has developed and programmed such methods based on the
multinomial logit model, together with optimal bandwidth selection and
bias correction techniques \textbf{(Xu, 2017)}.

Fourth, the robustness checks using donut hole regression discontinuity
models are limited considering the possible non-random heaping in the
data. The original study used donut hole models primarily to deal with
potential confounders, but non-random heaping might also exist in light
of the present study. Although the original study runs the model with
observations dropped within windows of different sizes and found
relatively consistent results, this method could still give biased
estimates if non-random heaping outside the windows was biasing the
estimation \textbf{(Barreca et al., 2016)}. Future studies can adopt a
potential solution suggested by previous research, that is, to separate
the analyses for non-heaped data and heap points (if any). Unbiased
estimates can be derived by taking a population-weighted average of the
two estimates based on the restricted sample sizes \textbf{(Barreca et
al., 2016)}.

Finally, the external and internal validity of the study can be improved
in future extensions. To improve the external validity of the
estimation, one can use data from regions other than the state of
Washington. This is to establish that punishments are universally
effective in reducing repeat drunk driving across regions and the effect
is not caused by some other factors specific to the state of Washington.
The internal validity of the causal relation can be improved by
conducting a longitudinal field experiment. One can choose a region
where laws criminalizing DUI are not yet enforced, and carry out an
experiment by manipulating the magnitude of the punishments drivers will
receive if found driving under the influence. Drivers with a BAC level
just below a chosen threshold are randomly assigned to receive lighter
punishments than those with a BAC level just above the threshold. This
method can theoretically provide stronger evidence for the causal
relation between receiving punishments and recidivism. However, there
are also challenges to such designs. For example, methodologically
speaking, it is hard to prevent spillover effect. Drivers may find out
punishments vary among individuals, and the consequence of such an
effect is essentially unpredictable. Ethically speaking, it also
requires debate on how such an experiment adhere to the principle of
transparency. Due to the scope of this paper, I will not expand further
on these topics and will leave them to be addressed by future research.

\hypertarget{conclusion}{%
\section{Conclusion}\label{conclusion}}

The present study successfully replicates the main results of the study
\emph{Punishment and Deterrence: Evidence from Drunk Driving}
\textbf{(Hansen, 2015)} about the effect of punishments and sanctions on
reducing repeat drunk driving using a regression discontinuity design.
Specifically, I find that receiving punishments at the 0.08 BAC
threshold reduces recidivism by up to 2.4 percentage points according to
local linear regression models, and the decrease is up to 1.6 percentage
points according to local second-order polynomials. These results
provide evidence for the validity of the original study. The original
study also goes beyond these estimates and reveals heterogeneity of the
effects. Despite these strengths, the study can still be improved in
aspects such as bandwidth selection, the selection of appropriate models
and robustness checks, the presentation of the results, as well as
internal and external validity.

\end{document}
