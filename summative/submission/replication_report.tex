% Options for packages loaded elsewhere
\PassOptionsToPackage{unicode}{hyperref}
\PassOptionsToPackage{hyphens}{url}
%
\documentclass[
  11pt,
]{article}
\usepackage{amsmath,amssymb}
\usepackage{lmodern}
\usepackage{iftex}
\ifPDFTeX
  \usepackage[T1]{fontenc}
  \usepackage[utf8]{inputenc}
  \usepackage{textcomp} % provide euro and other symbols
\else % if luatex or xetex
  \usepackage{unicode-math}
  \defaultfontfeatures{Scale=MatchLowercase}
  \defaultfontfeatures[\rmfamily]{Ligatures=TeX,Scale=1}
  \setmainfont[]{Times New Roman}
\fi
% Use upquote if available, for straight quotes in verbatim environments
\IfFileExists{upquote.sty}{\usepackage{upquote}}{}
\IfFileExists{microtype.sty}{% use microtype if available
  \usepackage[]{microtype}
  \UseMicrotypeSet[protrusion]{basicmath} % disable protrusion for tt fonts
}{}
\makeatletter
\@ifundefined{KOMAClassName}{% if non-KOMA class
  \IfFileExists{parskip.sty}{%
    \usepackage{parskip}
  }{% else
    \setlength{\parindent}{0pt}
    \setlength{\parskip}{6pt plus 2pt minus 1pt}}
}{% if KOMA class
  \KOMAoptions{parskip=half}}
\makeatother
\usepackage{xcolor}
\usepackage[margin=1.0in]{geometry}
\usepackage{graphicx}
\makeatletter
\def\maxwidth{\ifdim\Gin@nat@width>\linewidth\linewidth\else\Gin@nat@width\fi}
\def\maxheight{\ifdim\Gin@nat@height>\textheight\textheight\else\Gin@nat@height\fi}
\makeatother
% Scale images if necessary, so that they will not overflow the page
% margins by default, and it is still possible to overwrite the defaults
% using explicit options in \includegraphics[width, height, ...]{}
\setkeys{Gin}{width=\maxwidth,height=\maxheight,keepaspectratio}
% Set default figure placement to htbp
\makeatletter
\def\fps@figure{htbp}
\makeatother
\setlength{\emergencystretch}{3em} % prevent overfull lines
\providecommand{\tightlist}{%
  \setlength{\itemsep}{0pt}\setlength{\parskip}{0pt}}
\setcounter{secnumdepth}{5}
\newcommand{\bcenter}{\begin{center}}
\newcommand{\ecenter}{\end{center}}
\newcommand{\btitlepage}{\begin{titlepage}}
\newcommand{\etitlepage}{\end{titlepage}}
\usepackage{setspace}\onehalfspacing
\usepackage{booktabs}
\usepackage[font=small,labelfont=bf]{caption}
\ifLuaTeX
  \usepackage{selnolig}  % disable illegal ligatures
\fi
\IfFileExists{bookmark.sty}{\usepackage{bookmark}}{\usepackage{hyperref}}
\IfFileExists{xurl.sty}{\usepackage{xurl}}{} % add URL line breaks if available
\urlstyle{same} % disable monospaced font for URLs
\hypersetup{
  hidelinks,
  pdfcreator={LaTeX via pandoc}}

\author{}
\date{\vspace{-2.5em}}

\begin{document}

\begin{titlepage}

\begin{center}

\vspace*{30mm}

Candidate number: 49045

\vspace*{5mm}

\hypertarget{replication-of-hansen-2015-punishment-and-deterrence-evidence-from-drunk-driving}{%
\section*{Replication of Hansen (2015) Punishment and Deterrence:
Evidence from Drunk
Driving}\label{replication-of-hansen-2015-punishment-and-deterrence-evidence-from-drunk-driving}}
\addcontentsline{toc}{section}{Replication of Hansen (2015) Punishment
and Deterrence: Evidence from Drunk Driving}

\vspace*{30mm}

Submitted as the summative assessment for\\

PB4A7: Quantitative Applications for Behavioural Science 2022

\end{center}

\end{titlepage}

\newpage

In the study \emph{Punishment and Deterrence: Evidence from Drunk
Driving} \textbf{(Hansen, 2015)}, Hansen investigated the effect of
punishments and sanctions on reducing repeat drunk driving. The study
implemented a quasi-experimental design, utilising the administrative
records from 1995 to 2011 of the state of Washington, U.S., where two
thresholds of blood alcohol content (BAC) are used to determine the
status of driving under the influence (DUI). Specifically, a driver with
a measured BAC over 0.08 is considered a case of DUI and will be
punished via measures such as fines, jail time, and driving license
suspension. One with a BAC over 0.15 is considered a case of aggravated
DUI, to which more severe punishments and sanctions are applied. Since
BAC measure has clear-cut numeric thresholds for determining whether a
driver will receive harsher punishments, and neither drivers or police
can manipulate this measure, Hansen applied a regression discontinuity
design to analyse the data \textbf{(expand on the justification of using
RDD)}. He hypothesized that receiving harsher punishments and sanctions
at both thresholds would reduce offenders' future recidivism of DUI. He
further applied the RDD to analyse the effect of receiving harsher
punishments on the degree of deterrence, incapacitation, and
rehabilitation in order to identify the mechanism through which harsher
punishments might reduce recidivism \textbf{(check the paper to see
whether this sentence is precise)}.

\textbf{Ideally also the results}

The present study aims to replicate the findings from \textbf{Hansen
(2015)} using regression discontinuity design applied to a similar data
\textbf{(what is it exactly\ldots)}. It will be focused on estimating
the effect of receiving punishments on reducing recidivism at the 0.08
BAC threshold only. The paper proceeds as follows. Section 1 discusses
the econometric method and assumptions underlying its application to the
current data. Section 2 presents the main results, and Section 3
discusses critiques and extensions to the original study. Section 4
concludes.

\hypertarget{methods-and-assumptions}{%
\section{Methods and Assumptions}\label{methods-and-assumptions}}

The present study applies the regression discontinuity design to the
data provided by the class instructor. Several assumptions need to be
met so that the regression discontinuity design can give valid and
accurate estimates. First, people need to be randomly assigned to
receiving punishment at the BAC threshold. In other words, neither the
drivers nor the police can manipulate the BAC level and thus whether one
will receive punishment. \textbf{Hansen (2015)} has already given a
plausible theoretical justification for this assumption. Empirically, I
plot the distribution of BAC and test for discontinuity in its
distribution at the threshold. Figure \ref{fig:bac_hist_continuous}
shows the distribution of BAC level. Based on a recently developed local
polynomial density estimator \textbf{(CJM, 2020)}, the hypothesis that
the distribution is continuous at 0.08 cannot be rejected at the level
of 0.05 (\emph{p} = 0.890). Therefore, there is no evidence for the
existence of manipulation on the BAC level.

\begin{figure}[h]
  \centering
  \includegraphics[width=0.9\columnwidth]{../figures/bac_histogram_continuous.pdf}
  \caption{\textbf{Histogram of blood alcohol content as a continuous variable.} The blood alcohol content is plotted as a continuous variable with a bin width of 0.001, the precision used on the breathalysers. The y-axis represents the frequency of observations in each bin. The vertical black lines represent the two thresholds at 0.08 and 0.15.}
  \label{fig:bac_hist_continuous}
\end{figure}

The second assumption is that the running variable should not contain
non-random heaping, that is, BAC should not be much more likely to take
certain values than others. Based on figure
\ref{fig:bac_hist_continuous}, this assumption seems to be violated.
Curiously, when BAC is plotted as a discrete variable, non-random
heaping disappears (Figure \ref{fig:bac_hist_discrete}). This probably
originates from the lack of precision in BAC in the current data. Many
BAC values deviates by a very small amount from the values that are
supposed to be given by the breathalysers (i.e., precise to three digits
after the decimal point). For bins where the value at the left boundary
deviates upwards and the values at the right boundary deviates
downwards, they will contain a larger number of observations than other
binds and thus leads to heaping. This would not occur if the values were
precise so that the number of observations are rightly plotted by two
bins, or if we treat BAC as discrete and give each value its own bin.
\textbf{(consider using a simulation to justify this claim?)}

\begin{figure}[h]
  \centering
  \includegraphics[width=0.9\columnwidth]{../figures/bac_histogram_discrete.pdf}
  \caption{\textbf{Histogram of blood alcohol content as a discrete variable.} The blood alcohol content is plotted as a discrete variable with a bin width of 0.001, the precision used on the breathalysers. The y-axis represents the frequency of observations in each bin. The vertical black lines represent the two thresholds at 0.08 and 0.15.}
  \label{fig:bac_hist_discrete}
\end{figure}

\hypertarget{results}{%
\section{Results}\label{results}}

\hypertarget{critique-and-extension}{%
\section{Critique and Extension}\label{critique-and-extension}}

\hypertarget{conclusion}{%
\section{Conclusion}\label{conclusion}}

\end{document}
